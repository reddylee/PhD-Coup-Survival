% Options for packages loaded elsewhere
\PassOptionsToPackage{unicode}{hyperref}
\PassOptionsToPackage{hyphens}{url}
\PassOptionsToPackage{dvipsnames,svgnames,x11names}{xcolor}
%
\documentclass[
  12pt,
  a4paper,
  DIV=11,
  numbers=noendperiod]{scrartcl}

\usepackage{amsmath,amssymb}
\usepackage{setspace}
\usepackage{iftex}
\ifPDFTeX
  \usepackage[T1]{fontenc}
  \usepackage[utf8]{inputenc}
  \usepackage{textcomp} % provide euro and other symbols
\else % if luatex or xetex
  \usepackage{unicode-math}
  \defaultfontfeatures{Scale=MatchLowercase}
  \defaultfontfeatures[\rmfamily]{Ligatures=TeX,Scale=1}
\fi
\usepackage{lmodern}
\ifPDFTeX\else  
    % xetex/luatex font selection
\fi
% Use upquote if available, for straight quotes in verbatim environments
\IfFileExists{upquote.sty}{\usepackage{upquote}}{}
\IfFileExists{microtype.sty}{% use microtype if available
  \usepackage[]{microtype}
  \UseMicrotypeSet[protrusion]{basicmath} % disable protrusion for tt fonts
}{}
\usepackage{xcolor}
\setlength{\emergencystretch}{3em} % prevent overfull lines
\setcounter{secnumdepth}{5}
% Make \paragraph and \subparagraph free-standing
\ifx\paragraph\undefined\else
  \let\oldparagraph\paragraph
  \renewcommand{\paragraph}[1]{\oldparagraph{#1}\mbox{}}
\fi
\ifx\subparagraph\undefined\else
  \let\oldsubparagraph\subparagraph
  \renewcommand{\subparagraph}[1]{\oldsubparagraph{#1}\mbox{}}
\fi


\providecommand{\tightlist}{%
  \setlength{\itemsep}{0pt}\setlength{\parskip}{0pt}}\usepackage{longtable,booktabs,array}
\usepackage{calc} % for calculating minipage widths
% Correct order of tables after \paragraph or \subparagraph
\usepackage{etoolbox}
\makeatletter
\patchcmd\longtable{\par}{\if@noskipsec\mbox{}\fi\par}{}{}
\makeatother
% Allow footnotes in longtable head/foot
\IfFileExists{footnotehyper.sty}{\usepackage{footnotehyper}}{\usepackage{footnote}}
\makesavenoteenv{longtable}
\usepackage{graphicx}
\makeatletter
\def\maxwidth{\ifdim\Gin@nat@width>\linewidth\linewidth\else\Gin@nat@width\fi}
\def\maxheight{\ifdim\Gin@nat@height>\textheight\textheight\else\Gin@nat@height\fi}
\makeatother
% Scale images if necessary, so that they will not overflow the page
% margins by default, and it is still possible to overwrite the defaults
% using explicit options in \includegraphics[width, height, ...]{}
\setkeys{Gin}{width=\maxwidth,height=\maxheight,keepaspectratio}
% Set default figure placement to htbp
\makeatletter
\def\fps@figure{htbp}
\makeatother
\newlength{\cslhangindent}
\setlength{\cslhangindent}{1.5em}
\newlength{\csllabelwidth}
\setlength{\csllabelwidth}{3em}
\newlength{\cslentryspacingunit} % times entry-spacing
\setlength{\cslentryspacingunit}{\parskip}
\newenvironment{CSLReferences}[2] % #1 hanging-ident, #2 entry spacing
 {% don't indent paragraphs
  \setlength{\parindent}{0pt}
  % turn on hanging indent if param 1 is 1
  \ifodd #1
  \let\oldpar\par
  \def\par{\hangindent=\cslhangindent\oldpar}
  \fi
  % set entry spacing
  \setlength{\parskip}{#2\cslentryspacingunit}
 }%
 {}
\usepackage{calc}
\newcommand{\CSLBlock}[1]{#1\hfill\break}
\newcommand{\CSLLeftMargin}[1]{\parbox[t]{\csllabelwidth}{#1}}
\newcommand{\CSLRightInline}[1]{\parbox[t]{\linewidth - \csllabelwidth}{#1}\break}
\newcommand{\CSLIndent}[1]{\hspace{\cslhangindent}#1}

\KOMAoption{captions}{tableheading}
\makeatletter
\makeatother
\makeatletter
\makeatother
\makeatletter
\@ifpackageloaded{caption}{}{\usepackage{caption}}
\AtBeginDocument{%
\ifdefined\contentsname
  \renewcommand*\contentsname{Table of contents}
\else
  \newcommand\contentsname{Table of contents}
\fi
\ifdefined\listfigurename
  \renewcommand*\listfigurename{List of Figures}
\else
  \newcommand\listfigurename{List of Figures}
\fi
\ifdefined\listtablename
  \renewcommand*\listtablename{List of Tables}
\else
  \newcommand\listtablename{List of Tables}
\fi
\ifdefined\figurename
  \renewcommand*\figurename{Figure}
\else
  \newcommand\figurename{Figure}
\fi
\ifdefined\tablename
  \renewcommand*\tablename{Table}
\else
  \newcommand\tablename{Table}
\fi
}
\@ifpackageloaded{float}{}{\usepackage{float}}
\floatstyle{ruled}
\@ifundefined{c@chapter}{\newfloat{codelisting}{h}{lop}}{\newfloat{codelisting}{h}{lop}[chapter]}
\floatname{codelisting}{Listing}
\newcommand*\listoflistings{\listof{codelisting}{List of Listings}}
\makeatother
\makeatletter
\@ifpackageloaded{caption}{}{\usepackage{caption}}
\@ifpackageloaded{subcaption}{}{\usepackage{subcaption}}
\makeatother
\makeatletter
\@ifpackageloaded{tcolorbox}{}{\usepackage[skins,breakable]{tcolorbox}}
\makeatother
\makeatletter
\@ifundefined{shadecolor}{\definecolor{shadecolor}{rgb}{.97, .97, .97}}
\makeatother
\makeatletter
\makeatother
\makeatletter
\makeatother
\ifLuaTeX
  \usepackage{selnolig}  % disable illegal ligatures
\fi
\IfFileExists{bookmark.sty}{\usepackage{bookmark}}{\usepackage{hyperref}}
\IfFileExists{xurl.sty}{\usepackage{xurl}}{} % add URL line breaks if available
\urlstyle{same} % disable monospaced font for URLs
\hypersetup{
  pdftitle={Political Leadership Survival in the Aftermath of Coups or Overstays},
  pdfauthor={Zhu Qi},
  pdfkeywords={Political survival, Coups, Overstays},
  colorlinks=true,
  linkcolor={blue},
  filecolor={Maroon},
  citecolor={Blue},
  urlcolor={Blue},
  pdfcreator={LaTeX via pandoc}}

\title{Political Leadership Survival in the Aftermath of Coups or
Overstays}
\author{Zhu Qi}
\date{2023-11-01}

\begin{document}
\maketitle
\begin{abstract}
\textbf{Abstract:} This study endeavors to conduct an in-depth analysis
of the determinants impacting the survival of political leadership that
has ascended to power, either through coups or overstays. Utilizing a
survival model, the research investigates the influence of factors such
as political stability, military control, economic performance, external
alliances, regime types, and levels of democracy on the endurance of
political leaders. Employing a quantitative approach with a novel
dataset encompassing coups and overstays, this research makes a
significant contribution to existing literature by providing valuable
insights into the factors shaping political survival for leaders who
assume office or maintain power through unconstitutional means.
\end{abstract}
\ifdefined\Shaded\renewenvironment{Shaded}{\begin{tcolorbox}[enhanced, boxrule=0pt, breakable, borderline west={3pt}{0pt}{shadecolor}, interior hidden, frame hidden, sharp corners]}{\end{tcolorbox}}\fi

\setstretch{1.75}
\hypertarget{introduction}{%
\section{Introduction}\label{introduction}}

In political science, a compelling enigma persists: why do certain
leaders manage to cling to power for three or four decades, while others
find their tenures cut short after just several years, or even mere
months or days? This intriguing question has been extensively explored
in numerous existing works Bueno de Mesquita et al.
(\protect\hyperlink{ref-buenodemesquita2003}{2003}). Some of these works
encompass leaders across diverse political landscapes, spanning
democracies and autocracies, parliamentary and presidential systems, as
well as civilian and military contexts
(\protect\hyperlink{ref-buenodemesquita2003}{Bueno de Mesquita et al.
2003}). Others narrow their focus to specific types of regimes, delving
into democracies (\protect\hyperlink{ref-svolik2014}{Svolik 2014}) or
autocracies (\protect\hyperlink{ref-davenport2021}{Davenport,
RezaeeDaryakenari, and Wood 2021}).

However, a significant number of political leaders, particularly in
democracies and some in autocracies, undergo regular and predictable
tenures. A prime example is found in the United States, where presidents
may stay in the White House for up to eight years if they perform well
and secure a second term. Even in cases of poor performance, they
typically complete a full four-year term. Another illustration can be
drawn from autocratic Mexico between 1919 and 2000, where each president
served a fixed six-year term without facing overthrows or overstays. In
such scenarios, analyzing the survival of these leaders seems futile, as
power transitions from one political leader to the next are typically
observed within the framework of constitutional rules or unwritten
conventions.

The central emphasis on political longevity centers around leaders who
stay in power for unforeseen durations. In theory, such situations can
happen in any political context. Even in the United States, one of the
most regular power transition countries, President Trump, following his
electoral defeat, endeavoured to extend his tenure by contesting the
outcome of the general election. However, the predominant instances of
unexpected political tenures revolve around leaders who either seize
power through coups or overstay through unconstitutional means---this
constitutes the core focus of this paper.

The analysis of their tenures is particularly significant for two
reasons. Firstly, the durations of these leaders' tenures exhibit
considerable variation, ranging from mere months to several decades.
Secondly, predicting the tenures of such leaders proves challenging. A
seemingly robust and stable regime can collapse suddenly overnight,
while an apparently fragile one might persist for decades. The
substantial disparities in these tenures remain inadequately explained,
posing a perplexing challenge and attracting the attention of numerous
political scientists.

Building upon discussions surrounding coups and incumbent overstays,
this paper delves into the trajectories of political leaders who
ascended to power through coups or overstayed their intended terms. The
primary focus lies in unravelling the duration of these leaders' tenures
and understanding the underlying determinants.

\hypertarget{theories}{%
\section{Theories}\label{theories}}

The survival of political leaders following coups or overstays may hinge
on six pivotal factors:

\hypertarget{coups-vs.-overstays}{%
\subsection{Coups vs.~overstays}\label{coups-vs.-overstays}}

Survival in power relies significantly on the cohesion of the ruling
group. As numerous scholars have pointed out, internal conflicts among
elites pose a more serious threat to the stability of those in power.
Coups often lay bare the fractures within a regime, not only attracting
more followers to orchestrate new coups but also emboldening external
challengers, including uprisings, revolutions, and civil wars. On the
other hand, successful tenures unmistakably showcase the incumbents'
firm grasp on power, discouraging both internal dissent and external
threats (\protect\hyperlink{ref-dahl2023}{Dahl and Gleditsch 2023}).

\textbf{Hypothesis 1 (H1):} Political leaders who successfully extend
their time in power are more likely to have prolonged survival compared
to leaders who assume power through coups.

\hypertarget{regime-types}{%
\subsection{Regime types}\label{regime-types}}

In the majority of cases, regimes following coups or prolonged stays
tend to be non-democratic. Democratic leaders are generally anticipated
to relinquish power in a regular and cyclical manner. Conversely, for
non-democratic leaders, the duration of their tenures is heavily
influenced by the type of autocracy. The three primary autocratic
regimes are dominant party, military, and personal.

Within the military regime, leaders often encounter more challenges
during their tenures. The ability to challenge incumbents, particularly
those within ruling groups, relies significantly on the support of
military forces. In dominant party or personal regimes, the military
typically operates under the control of party or personal leaders, who
are the incumbents themselves. Unlike military regimes, where generals
often play significant roles in politics, there are typically many
generals in dominant party or personal regimes, acting as checks and
balances on each other. Military regimes, however, with their powerful
army leaders and more influential generals, are more prone to political
interference and internal conflicts, leading to shorter tenures for
leaders in such regimes.

Hypothesis 2 (H2): Leaders in dominant party or personal regimes are
expected to have longer survival periods than those in military regimes.

\hypertarget{societal-stability}{%
\subsection{Societal stability}\label{societal-stability}}

\textbf{Hypothesis 3 (H3):} Political leaders presiding over stable
societies are likely to experience longer tenures.

\hypertarget{purges-and-repressions}{%
\subsection{Purges and repressions}\label{purges-and-repressions}}

\textbf{Hypothesis 4 (H4):} Leaders who are more prone to employ
stringent repression against dissidents are expected to have longer
survival durations.

\hypertarget{external-alliances}{%
\subsection{External alliances}\label{external-alliances}}

\textbf{Hypothesis 5 (H5):} Leaders with strong external alliances are
anticipated to have extended survival periods.

\hypertarget{economic-performance}{%
\subsection{Economic performance}\label{economic-performance}}

\textbf{Hypothesis 6 (H6):} Leaders with a robust economic performance
are likely to endure longer than their counterparts facing economic
crises.

\newpage

\hypertarget{references}{%
\section*{References}\label{references}}
\addcontentsline{toc}{section}{References}

\hypertarget{refs}{}
\begin{CSLReferences}{1}{0}
\leavevmode\vadjust pre{\hypertarget{ref-buenodemesquita2003}{}}%
Bueno de Mesquita, Bruce, Alastair Smith, Randolph M. Siverson, and
James D. Morrow. 2003. \emph{The Logic of Political Survival}. The MIT
Press. \url{https://doi.org/10.7551/mitpress/4292.001.0001}.

\leavevmode\vadjust pre{\hypertarget{ref-clinton1975politics}{}}%
Clinton, Richard Lee. 1975. {``Politics and Survival.''} \emph{World
Affs.} 138: 108.

\leavevmode\vadjust pre{\hypertarget{ref-dahl2023}{}}%
Dahl, Marianne, and Kristian Skrede Gleditsch. 2023. {``Clouds with
Silver Linings: How Mobilization Shapes the Impact of Coups on
Democratization.''} \emph{European Journal of International Relations},
January, 135406612211432.
\url{https://doi.org/10.1177/13540661221143213}.

\leavevmode\vadjust pre{\hypertarget{ref-davenport2021}{}}%
Davenport, Christian, Babak RezaeeDaryakenari, and Reed M Wood. 2021.
{``Tenure Through Tyranny? Repression, Dissent, and Leader Removal in
Africa and Latin America, 1990{\textendash}2006.''} \emph{Journal of
Global Security Studies} 7 (1).
\url{https://doi.org/10.1093/jogss/ogab023}.

\leavevmode\vadjust pre{\hypertarget{ref-svolik2014}{}}%
Svolik, Milan W. 2014. {``Which Democracies Will Last? Coups, Incumbent
Takeovers, and the Dynamic of Democratic Consolidation.''} \emph{British
Journal of Political Science} 45 (4): 715--38.
\url{https://doi.org/10.1017/s0007123413000550}.

\end{CSLReferences}



\end{document}
