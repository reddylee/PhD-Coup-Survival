% Options for packages loaded elsewhere
\PassOptionsToPackage{unicode}{hyperref}
\PassOptionsToPackage{hyphens}{url}
\PassOptionsToPackage{dvipsnames,svgnames,x11names}{xcolor}
%
\documentclass[
  12pt,
  a4paper,
  12pt]{article}

\usepackage{amsmath,amssymb}
\usepackage{setspace}
\usepackage{iftex}
\ifPDFTeX
  \usepackage[T1]{fontenc}
  \usepackage[utf8]{inputenc}
  \usepackage{textcomp} % provide euro and other symbols
\else % if luatex or xetex
  \usepackage{unicode-math}
  \defaultfontfeatures{Scale=MatchLowercase}
  \defaultfontfeatures[\rmfamily]{Ligatures=TeX,Scale=1}
\fi
\usepackage{lmodern}
\ifPDFTeX\else  
    % xetex/luatex font selection
  \setmainfont[]{Times New Roman}
\fi
% Use upquote if available, for straight quotes in verbatim environments
\IfFileExists{upquote.sty}{\usepackage{upquote}}{}
\IfFileExists{microtype.sty}{% use microtype if available
  \usepackage[]{microtype}
  \UseMicrotypeSet[protrusion]{basicmath} % disable protrusion for tt fonts
}{}
\usepackage{xcolor}
\setlength{\emergencystretch}{3em} % prevent overfull lines
\setcounter{secnumdepth}{5}
% Make \paragraph and \subparagraph free-standing
\ifx\paragraph\undefined\else
  \let\oldparagraph\paragraph
  \renewcommand{\paragraph}[1]{\oldparagraph{#1}\mbox{}}
\fi
\ifx\subparagraph\undefined\else
  \let\oldsubparagraph\subparagraph
  \renewcommand{\subparagraph}[1]{\oldsubparagraph{#1}\mbox{}}
\fi


\providecommand{\tightlist}{%
  \setlength{\itemsep}{0pt}\setlength{\parskip}{0pt}}\usepackage{longtable,booktabs,array}
\usepackage{calc} % for calculating minipage widths
% Correct order of tables after \paragraph or \subparagraph
\usepackage{etoolbox}
\makeatletter
\patchcmd\longtable{\par}{\if@noskipsec\mbox{}\fi\par}{}{}
\makeatother
% Allow footnotes in longtable head/foot
\IfFileExists{footnotehyper.sty}{\usepackage{footnotehyper}}{\usepackage{footnote}}
\makesavenoteenv{longtable}
\usepackage{graphicx}
\makeatletter
\def\maxwidth{\ifdim\Gin@nat@width>\linewidth\linewidth\else\Gin@nat@width\fi}
\def\maxheight{\ifdim\Gin@nat@height>\textheight\textheight\else\Gin@nat@height\fi}
\makeatother
% Scale images if necessary, so that they will not overflow the page
% margins by default, and it is still possible to overwrite the defaults
% using explicit options in \includegraphics[width, height, ...]{}
\setkeys{Gin}{width=\maxwidth,height=\maxheight,keepaspectratio}
% Set default figure placement to htbp
\makeatletter
\def\fps@figure{htbp}
\makeatother

\addtolength{\oddsidemargin}{-.5in}%
\addtolength{\evensidemargin}{-1in}%
\addtolength{\textwidth}{1in}%
\addtolength{\textheight}{1.7in}%
\addtolength{\topmargin}{-1in}%
\makeatletter
\makeatother
\makeatletter
\makeatother
\makeatletter
\@ifpackageloaded{caption}{}{\usepackage{caption}}
\AtBeginDocument{%
\ifdefined\contentsname
  \renewcommand*\contentsname{Table of contents}
\else
  \newcommand\contentsname{Table of contents}
\fi
\ifdefined\listfigurename
  \renewcommand*\listfigurename{List of Figures}
\else
  \newcommand\listfigurename{List of Figures}
\fi
\ifdefined\listtablename
  \renewcommand*\listtablename{List of Tables}
\else
  \newcommand\listtablename{List of Tables}
\fi
\ifdefined\figurename
  \renewcommand*\figurename{Figure}
\else
  \newcommand\figurename{Figure}
\fi
\ifdefined\tablename
  \renewcommand*\tablename{Table}
\else
  \newcommand\tablename{Table}
\fi
}
\@ifpackageloaded{float}{}{\usepackage{float}}
\floatstyle{ruled}
\@ifundefined{c@chapter}{\newfloat{codelisting}{h}{lop}}{\newfloat{codelisting}{h}{lop}[chapter]}
\floatname{codelisting}{Listing}
\newcommand*\listoflistings{\listof{codelisting}{List of Listings}}
\makeatother
\makeatletter
\@ifpackageloaded{caption}{}{\usepackage{caption}}
\@ifpackageloaded{subcaption}{}{\usepackage{subcaption}}
\makeatother
\makeatletter
\@ifpackageloaded{tcolorbox}{}{\usepackage[skins,breakable]{tcolorbox}}
\makeatother
\makeatletter
\@ifundefined{shadecolor}{\definecolor{shadecolor}{rgb}{.97, .97, .97}}
\makeatother
\makeatletter
\makeatother
\makeatletter
\makeatother
\ifLuaTeX
  \usepackage{selnolig}  % disable illegal ligatures
\fi
\usepackage[]{natbib}
\bibliographystyle{agsm}
\IfFileExists{bookmark.sty}{\usepackage{bookmark}}{\usepackage{hyperref}}
\IfFileExists{xurl.sty}{\usepackage{xurl}}{} % add URL line breaks if available
\urlstyle{same} % disable monospaced font for URLs
\hypersetup{
  pdftitle={Political Leadership Survival in the Aftermath of Coups or Overstays: From Illegitimate Ascent to Unexpected Exit},
  pdfauthor={Zhu Qi},
  pdfkeywords={Political survival, Coups, Overstays},
  colorlinks=true,
  linkcolor={blue},
  filecolor={Maroon},
  citecolor={Blue},
  urlcolor={Blue},
  pdfcreator={LaTeX via pandoc}}


\begin{document}


\def\spacingset#1{\renewcommand{\baselinestretch}%
{#1}\small\normalsize} \spacingset{1}


%%%%%%%%%%%%%%%%%%%%%%%%%%%%%%%%%%%%%%%%%%%%%%%%%%%%%%%%%%%%%%%%%%%%%%%%%%%%%%

\date{November 4, 2023}
\title{\bf Political Leadership Survival in the Aftermath of Coups or
Overstays: From Illegitimate Ascent to Unexpected Exit}
\author{
Zhu Qi\\
Department of Government, University of Essex\\
}
\maketitle

\bigskip
\bigskip
\begin{abstract}
Previous research predominantly focused on the disruption of regular
leadership tenures by unexpected events, such as coups, self-coups,
uprisings, rebellions, civil wars, or revolutions. In contrast, this
study aims to delve into the longevity of leaders who ascend to power
through these very unexpected events, specifically coups or overstays.
The central argument posits that the endurance of political leadership
is influenced not only by their actions and policies in office but also
by the means through which they come to power. Employing a survival
model, this research investigates the disparities in survival rates
between leaders who rise to power via coups and those who overstay their
terms, and seeks to elucidate the underlying reasons for these
differences.
\end{abstract}

\noindent%
{\it Keywords:} Political survival, Coups, Overstays
\vfill

\newpage
\spacingset{1.9} % DON'T change the spacing!
\ifdefined\Shaded\renewenvironment{Shaded}{\begin{tcolorbox}[boxrule=0pt, sharp corners, frame hidden, interior hidden, borderline west={3pt}{0pt}{shadecolor}, enhanced, breakable]}{\end{tcolorbox}}\fi

\setstretch{1.75}
\hypertarget{introduction}{%
\section{Introduction}\label{introduction}}

The question of why some leaders maintain their hold on power for
extended periods, spanning decades, while others witness their
leadership cut short after mere years, months, or even days, has
captivated scholars and researchers in the field of political science.
This inquiry has been extensively explored in numerous works, as
evidenced by notable contributions such as those by
\citet{clinton1975politics} and \citet{buenodemesquita2003}.

In their seminal work, Bueno de Mesquita and his colleagues undertake a
comprehensive examination of leaders across a diverse political
landscape, encompassing democracies and autocracies, parliamentary and
presidential systems, and both civilian and military contexts. However,
it is worth noting that a significant number of political leaders,
especially in democratic countries, adhere to regular and predictable
tenures. An illustrative example can be found in the United States,
where presidents may serve up to eight years if they secure a second
term, even in cases of suboptimal performance. Similarly, in autocratic
Mexico from 1919 to 2000, each president served a fixed six-year term
without facing overthrows or overstays. In such contexts, the
investigation of tenure length is of marginal significance, as power
transitions between leaders typically occur within the established
framework of constitutional rules or unwritten conventions.

Given the distinctive nature of political leader survival in different
regimes, scholars have increasingly focused on the unexpected tenures,
namely those leaders who do not complete their original terms or those
who overstay their mandates. This shift in focus stems from the fact
that some leaders are toppled by coups, uprisings, rebellions, civil
wars, or revolutions, while others successfully navigate lawful or
unlawful challenges. Previous research on the longevity of political
leaders predominantly centers on two primary dimensions. The first
dimension encompasses the contextual conditions and resources available
to leaders, including factors such as their personal competence
\citep{yu2016}, the stability of their society \citep{arriola2009},
economic performance \citep[\citet{williams2011}]{palmer1999}, access to
natural resources \citep{smith2004, quirozflores2012}, and external
support networks \citep[\citet{thyne2017}]{licht2009, wright2008}. The
second dimension delves into the strategies employed by leaders in
enacting their political and economic policies
\citep{gandhi2007, morrison2009}, as well as their responses to
challenges and dissent within their regimes
\citep{escribà-folch2013, davenport2021}.

Unsurprisingly, a substantial portion of the existing research on
political survival predominantly centers on coups, as they represent the
most common pathways to the exit of authoritarian leaders
\citep{svolik2008, frantz2016}. Previous literature has primarily delved
into the survival of leadership in terms of strategies aimed at
preventing coups \citep{powell2017, sudduth2017, debruin2020}, or how
leaders can extend their tenures after surviving failed coup attempts
\citep{easton2018}.

However, on one hand, the duration of political leaders' tenures can be
significantly influenced by unforeseen events like coups. On the other
hand, these very unexpected events that usher in new leaders can also
become the catalyst for the subsequent cycle of unexpected developments.
It is conceivable that leaders who come to power through regular and
constitutional transitions are more likely to undergo periodic shifts in
leadership, while those who seize power through unconstitutional means
face a higher risk of unanticipated removal from office. Unfortunately,
there has been a limited emphasis on the study of leadership survival in
the context of successful coups. A similar research gap exists in the
examination of incumbents who overstay their terms in power, which forms
the central focus of this paper.

The analysis of their tenures holds particular significance for two
reasons. Firstly, the duration of these leaders' tenures exhibits
considerable variation, ranging from mere months to several decades.
Secondly, predicting the tenures of such leaders proves challenging. A
seemingly robust and stable regime can collapse suddenly overnight,
while an apparently fragile one might persist for decades. The
substantial disparities in these tenures remain inadequately explained,
posing a perplexing challenge that has piqued the interest of numerous
political scientists.

Expanding on the discourse surrounding coups and leaders who overstay
their intended terms, this paper delves into the trajectories of
political leaders who came to power through coups or extended their
mandates beyond the originally intended tenure. The central objective is
to examine the variations in survival duration between leaders who
attain power through coups and those who exceed their terms, while also
shedding light on the underlying factors contributing to these
distinctions.

This paper follows a structured approach as outlined below: The second
section encompasses a comprehensive literature review on political
survival and highlighting the contributions of this paper might offer.
The third chapter delves into the examination of factors influencing the
survival of leaders who have ascended to power through unconstitutional
means. Chapter 4 provides an account of the methodology and data
employed, utilizing a survival model for a comprehensive analysis of the
determinants of leaders' survival. The subsequent chapter, Chapter 5,
presents the findings of this analysis, facilitating an in-depth
discussion of the results. Finally, in Chapter 6, the paper concludes by
synthesizing these findings and exploring their broader implications.

\hypertarget{literature-review}{%
\section{Literature review}\label{literature-review}}

The duration for which political leaders can maintain their hold on
power is, to a significant extent, influenced by the manner in which
they ascended to leadership positions. It's conceivable that leaders who
engage in regular and constitutional transitions of power are more
likely to voluntarily step down as their terms expire, while those who
came to power through unconstitutional means are at a higher risk of
being unexpectedly removed from office. Much of the existing research on
the relationship between coups and the survival of leadership primarily
focuses on strategies to prevent coups
\citep{powell2017, sudduth2017, debruin2020} or how leaders can prolong
their stay in power after surviving failed coup attempts
\citep{easton2018}. Unfortunately, there is a limited emphasis on the
study of leadership survival in the context of successful coups. A
similar gap in research attention exists concerning incumbents who
overstay their terms in power.

In their seminal work, \citet{buenodemesquita2003} introduce and expound
upon the selectorate theory of politics. This theory centers on the
analysis of leadership survival based on the concept of a sufficiently
large winning coalition (\(W\)) within the selectorate (\(S\)). The
selectorate, which encompasses individuals with the authority to
determine leadership, contrasts with the winning coalition, signifying
the minimum number of selectorate members required to secure power. In
this framework, the endurance of political leaders depends on the
maintenance of a supportive winning coalition. Winning coalitions,
driven by the pursuit of benefits, opt to back incumbents, but their
allegiance may shift towards challengers if they anticipate greater
advantages from a change in leadership.

However, two critical issues arise within this framework. Firstly, in
democracies, while those who support and vote for incumbents may see
their preferred policies enacted, those who vote against them still face
the same policies. For example, individuals casting their votes for a
candidate in favor of lower taxes confront the same tax rates as those
who vote against the incumbents. This doesn't translate into lower taxes
for supporters and higher taxes for opponents; rather, both groups face
identical tax levels. Consequently, we cannot assert that winning
coalitions inherently gain a significant advantage over the broader
selectorate. Secondly, in many autocratic regimes, the process of
leadership selection remains shrouded in secrecy. In countries like
China, the mechanisms for appointing leaders resemble a black box, with
outsiders left unaware of the rules and procedures. Expressing
dissenting views, whether as potential challengers or supporters of
challengers, is fraught with danger. In Russia, despite the presence of
general elections, challengers often face perilous consequences,
including assassination, poisoning, imprisonment, or exile.

\hypertarget{theories}{%
\section{Theories}\label{theories}}

The survival of political leaders following coups or overstays may hinge
on six pivotal factors:

\hypertarget{coups-vs.-overstays}{%
\subsection{Coups vs.~overstays}\label{coups-vs.-overstays}}

Survival in power relies significantly on the cohesion of the ruling
group. As numerous scholars have pointed out, internal conflicts among
elites pose a more serious threat to the stability of those in power.
Coups often lay bare the fractures within a regime, not only attracting
more followers to orchestrate new coups but also emboldening external
challengers, including uprisings, revolutions, and civil wars. On the
other hand, successful tenures unmistakably showcase the incumbents'
firm grasp on power, discouraging both internal dissent and external
threats \citep{dahl2023}.

\textbf{Hypothesis 1 (H1):} Political leaders who successfully extend
their time in power are more likely to have prolonged survival compared
to leaders who assume power through coups.

\hypertarget{regime-types}{%
\subsection{Regime types}\label{regime-types}}

In the majority of cases, regimes following coups or prolonged stays
tend to be non-democratic. Democratic leaders are generally anticipated
to relinquish power in a regular and cyclical manner. Conversely, for
non-democratic leaders, the duration of their tenures is heavily
influenced by the type of autocracy. The three primary autocratic
regimes are dominant party, military, and personal.

Within the military regime, leaders often encounter more challenges
during their tenures. The ability to challenge incumbents, particularly
those within ruling groups, relies significantly on the support of
military forces. In dominant party or personal regimes, the military
typically operates under the control of party or personal leaders, who
are the incumbents themselves. Unlike military regimes, where generals
often play significant roles in politics, there are typically many
generals in dominant party or personal regimes, acting as checks and
balances on each other. Military regimes, however, with their powerful
army leaders and more influential generals, are more prone to political
interference and internal conflicts, leading to shorter tenures for
leaders in such regimes.

Hypothesis 2 (H2): Leaders in dominant party or personal regimes are
expected to have longer survival periods than those in military regimes.

\hypertarget{societal-stability}{%
\subsection{Societal stability}\label{societal-stability}}

\textbf{Hypothesis 3 (H3):} Political leaders presiding over stable
societies are likely to experience longer tenures.

\hypertarget{purges-and-repressions}{%
\subsection{Purges and repressions}\label{purges-and-repressions}}

\textbf{Hypothesis 4 (H4):} Leaders who are more prone to employ
stringent repression against dissidents are expected to have longer
survival durations.

\hypertarget{external-alliances}{%
\subsection{External alliances}\label{external-alliances}}

\textbf{Hypothesis 5 (H5):} Leaders with strong external alliances are
anticipated to have extended survival periods.

\hypertarget{economic-performance}{%
\subsection{Economic performance}\label{economic-performance}}

\textbf{Hypothesis 6 (H6):} Leaders with a robust economic performance
are likely to endure longer than their counterparts facing economic
crises.

\newpage


\renewcommand\refname{References}
  \bibliography{survival.bib}


\end{document}
