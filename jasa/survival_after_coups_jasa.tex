% Options for packages loaded elsewhere
\PassOptionsToPackage{unicode}{hyperref}
\PassOptionsToPackage{hyphens}{url}
\PassOptionsToPackage{dvipsnames,svgnames,x11names}{xcolor}
%
\documentclass[
  a4paper,
  12pt]{article}

\usepackage{amsmath,amssymb}
\usepackage{setspace}
\usepackage{iftex}
\ifPDFTeX
  \usepackage[T1]{fontenc}
  \usepackage[utf8]{inputenc}
  \usepackage{textcomp} % provide euro and other symbols
\else % if luatex or xetex
  \usepackage{unicode-math}
  \defaultfontfeatures{Scale=MatchLowercase}
  \defaultfontfeatures[\rmfamily]{Ligatures=TeX,Scale=1}
\fi
\usepackage{lmodern}
\ifPDFTeX\else  
    % xetex/luatex font selection
  \setmainfont[]{Source Serif Pro}
\fi
% Use upquote if available, for straight quotes in verbatim environments
\IfFileExists{upquote.sty}{\usepackage{upquote}}{}
\IfFileExists{microtype.sty}{% use microtype if available
  \usepackage[]{microtype}
  \UseMicrotypeSet[protrusion]{basicmath} % disable protrusion for tt fonts
}{}
\usepackage{xcolor}
\setlength{\emergencystretch}{3em} % prevent overfull lines
\setcounter{secnumdepth}{5}
% Make \paragraph and \subparagraph free-standing
\ifx\paragraph\undefined\else
  \let\oldparagraph\paragraph
  \renewcommand{\paragraph}[1]{\oldparagraph{#1}\mbox{}}
\fi
\ifx\subparagraph\undefined\else
  \let\oldsubparagraph\subparagraph
  \renewcommand{\subparagraph}[1]{\oldsubparagraph{#1}\mbox{}}
\fi


\providecommand{\tightlist}{%
  \setlength{\itemsep}{0pt}\setlength{\parskip}{0pt}}\usepackage{longtable,booktabs,array}
\usepackage{calc} % for calculating minipage widths
% Correct order of tables after \paragraph or \subparagraph
\usepackage{etoolbox}
\makeatletter
\patchcmd\longtable{\par}{\if@noskipsec\mbox{}\fi\par}{}{}
\makeatother
% Allow footnotes in longtable head/foot
\IfFileExists{footnotehyper.sty}{\usepackage{footnotehyper}}{\usepackage{footnote}}
\makesavenoteenv{longtable}
\usepackage{graphicx}
\makeatletter
\def\maxwidth{\ifdim\Gin@nat@width>\linewidth\linewidth\else\Gin@nat@width\fi}
\def\maxheight{\ifdim\Gin@nat@height>\textheight\textheight\else\Gin@nat@height\fi}
\makeatother
% Scale images if necessary, so that they will not overflow the page
% margins by default, and it is still possible to overwrite the defaults
% using explicit options in \includegraphics[width, height, ...]{}
\setkeys{Gin}{width=\maxwidth,height=\maxheight,keepaspectratio}
% Set default figure placement to htbp
\makeatletter
\def\fps@figure{htbp}
\makeatother

\addtolength{\oddsidemargin}{-.5in}%
\addtolength{\evensidemargin}{-1in}%
\addtolength{\textwidth}{1in}%
\addtolength{\textheight}{1.7in}%
\addtolength{\topmargin}{-1in}%
\makeatletter
\makeatother
\makeatletter
\makeatother
\makeatletter
\@ifpackageloaded{caption}{}{\usepackage{caption}}
\AtBeginDocument{%
\ifdefined\contentsname
  \renewcommand*\contentsname{Table of contents}
\else
  \newcommand\contentsname{Table of contents}
\fi
\ifdefined\listfigurename
  \renewcommand*\listfigurename{List of Figures}
\else
  \newcommand\listfigurename{List of Figures}
\fi
\ifdefined\listtablename
  \renewcommand*\listtablename{List of Tables}
\else
  \newcommand\listtablename{List of Tables}
\fi
\ifdefined\figurename
  \renewcommand*\figurename{Figure}
\else
  \newcommand\figurename{Figure}
\fi
\ifdefined\tablename
  \renewcommand*\tablename{Table}
\else
  \newcommand\tablename{Table}
\fi
}
\@ifpackageloaded{float}{}{\usepackage{float}}
\floatstyle{ruled}
\@ifundefined{c@chapter}{\newfloat{codelisting}{h}{lop}}{\newfloat{codelisting}{h}{lop}[chapter]}
\floatname{codelisting}{Listing}
\newcommand*\listoflistings{\listof{codelisting}{List of Listings}}
\makeatother
\makeatletter
\@ifpackageloaded{caption}{}{\usepackage{caption}}
\@ifpackageloaded{subcaption}{}{\usepackage{subcaption}}
\makeatother
\makeatletter
\makeatother
\ifLuaTeX
  \usepackage{selnolig}  % disable illegal ligatures
\fi
\usepackage[]{natbib}
\bibliographystyle{agsm}
\IfFileExists{bookmark.sty}{\usepackage{bookmark}}{\usepackage{hyperref}}
\IfFileExists{xurl.sty}{\usepackage{xurl}}{} % add URL line breaks if available
\urlstyle{same} % disable monospaced font for URLs
\hypersetup{
  pdftitle={Political Leadership Survival in the Aftermath of Coups or Overstays},
  pdfauthor={Zhu Qi},
  pdfkeywords={Political survival, Coups, Overstays},
  colorlinks=true,
  linkcolor={blue},
  filecolor={Maroon},
  citecolor={Blue},
  urlcolor={Blue},
  pdfcreator={LaTeX via pandoc}}


\begin{document}


\def\spacingset#1{\renewcommand{\baselinestretch}%
{#1}\small\normalsize} \spacingset{1}


%%%%%%%%%%%%%%%%%%%%%%%%%%%%%%%%%%%%%%%%%%%%%%%%%%%%%%%%%%%%%%%%%%%%%%%%%%%%%%

\date{October 9, 2023}
\title{\bf Political Leadership Survival in the Aftermath of Coups or
Overstays}
\author{
Zhu Qi\thanks{The authors gratefully acknowledge \emph{please remember
to list all relevant funding sources in the non-anonymized (unblinded)
version}.}\\
Department of Government, University of Essex\\
}
\maketitle

\bigskip
\bigskip
\begin{abstract}
This study endeavors to conduct an in-depth analysis of the determinants
impacting the survival of political leadership that has ascended to
power, either through coups or overstays. Utilizing a survival model,
the research investigates the influence of factors such as political
stability, military control, economic performance, external alliances,
regime types, and levels of democracy on the endurance of political
leaders. Employing a quantitative approach with a novel dataset
encompassing coups and overstays, this research makes a significant
contribution to existing literature by providing valuable insights into
the factors shaping political survival for leaders who assume office or
maintain power through unconstitutional means.
\end{abstract}

\noindent%
{\it Keywords:} Political survival, Coups, Overstays
\vfill

\newpage
\spacingset{1.9} % DON'T change the spacing!
\setstretch{1.75}
\section{Introduction}\label{introduction}

Building upon discussions surrounding coups and incumbent overstays,
this paper delves into the trajectories of political leaders who
ascended to power through coups or overstayed their intended terms. The
primary focus lies in unraveling the duration of these leaders' tenures
and understanding the underlying determinants.

Within the realm of political science, a prominent puzzle persists: why
do some leaders maintain their grip on power for three or four decades,
whereas the majority experience shorter tenures, with some enduring for
mere months or even days? Numerous existing works
\citep[\citet{buenodemesquita2003}]{clinton1975politics} have explored
this question in broad strokes. In contrast, this paper specifically
addresses leaders who acquire and sustain power through unconstitutional
means. Firstly, these leaders constitute the majority of unanticipated
instances of prolonged rule. Unlike those who ascend through
conventional means, whose tenures are generally predictable unless
disrupted by coups or managed overstay, the tenure of leaders with
unconventional paths varies significantly---from mere months to several
decades. The substantial variances in their tenures remain inadequately
explained, perplexing and attracting many more political scientists.

\section{Theories}\label{theories}

The survival of political leaders following coups or overstays may hinge
on six pivotal factors:

\subsection{Coups vs.~overstays}\label{coups-vs.-overstays}

Survival in power relies significantly on the cohesion of the ruling
group. As numerous scholars have pointed out, internal conflicts among
elites pose a more serious threat to the stability of those in power.
Coups often lay bare the fractures within a regime, not only attracting
more followers to orchestrate new coups but also emboldening external
challengers, including uprisings, revolutions, and civil wars. On the
other hand, successful tenures unmistakably showcase the incumbents'
firm grasp on power, discouraging both internal dissent and external
threats \citep{dahl2023}.

\textbf{Hypothesis 1 (H1):} Political leaders who successfully extend
their time in power are more likely to have prolonged survival compared
to leaders who assume power through coups.

\subsection{Economic performance}\label{economic-performance}

\textbf{Hypothesis 2 (H2):} Leaders with a robust economic performance
are likely to endure longer than their counterparts facing economic
crises.

\subsection{Regime types}\label{regime-types}

\textbf{Hypothesis 3 (H3):} Leaders who effectively control their armed
forces are expected to have longer survival periods than those with
powerful and independent military officers.

\subsection{External alliances}\label{external-alliances}

\textbf{Hypothesis 4 (H4):} Leaders with strong external alliances are
anticipated to have extended survival periods.

\subsection{Societal stablity}\label{societal-stablity}

\textbf{Hypothesis 5 (H5):} Political leaders presiding over stable
societies are likely to experience longer tenures.

\subsection{Repression level}\label{repression-level}

\textbf{Hypothesis 6 (H6):} Leaders who are more prone to employ
stringent repression against dissidents are expected to have longer
survival durations.

\newpage


\renewcommand\refname{References}
  \bibliography{survival.bib}


\end{document}
