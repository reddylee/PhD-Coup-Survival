% Options for packages loaded elsewhere
\PassOptionsToPackage{unicode}{hyperref}
\PassOptionsToPackage{hyphens}{url}
\PassOptionsToPackage{dvipsnames,svgnames,x11names}{xcolor}
%
\documentclass[
  12pt,
  a4paper,
  12pt]{article}

\usepackage{amsmath,amssymb}
\usepackage{setspace}
\usepackage{iftex}
\ifPDFTeX
  \usepackage[T1]{fontenc}
  \usepackage[utf8]{inputenc}
  \usepackage{textcomp} % provide euro and other symbols
\else % if luatex or xetex
  \usepackage{unicode-math}
  \defaultfontfeatures{Scale=MatchLowercase}
  \defaultfontfeatures[\rmfamily]{Ligatures=TeX,Scale=1}
\fi
\usepackage{lmodern}
\ifPDFTeX\else  
    % xetex/luatex font selection
  \setmainfont[]{Times New Roman}
\fi
% Use upquote if available, for straight quotes in verbatim environments
\IfFileExists{upquote.sty}{\usepackage{upquote}}{}
\IfFileExists{microtype.sty}{% use microtype if available
  \usepackage[]{microtype}
  \UseMicrotypeSet[protrusion]{basicmath} % disable protrusion for tt fonts
}{}
\usepackage{xcolor}
\setlength{\emergencystretch}{3em} % prevent overfull lines
\setcounter{secnumdepth}{5}
% Make \paragraph and \subparagraph free-standing
\ifx\paragraph\undefined\else
  \let\oldparagraph\paragraph
  \renewcommand{\paragraph}[1]{\oldparagraph{#1}\mbox{}}
\fi
\ifx\subparagraph\undefined\else
  \let\oldsubparagraph\subparagraph
  \renewcommand{\subparagraph}[1]{\oldsubparagraph{#1}\mbox{}}
\fi


\providecommand{\tightlist}{%
  \setlength{\itemsep}{0pt}\setlength{\parskip}{0pt}}\usepackage{longtable,booktabs,array}
\usepackage{calc} % for calculating minipage widths
% Correct order of tables after \paragraph or \subparagraph
\usepackage{etoolbox}
\makeatletter
\patchcmd\longtable{\par}{\if@noskipsec\mbox{}\fi\par}{}{}
\makeatother
% Allow footnotes in longtable head/foot
\IfFileExists{footnotehyper.sty}{\usepackage{footnotehyper}}{\usepackage{footnote}}
\makesavenoteenv{longtable}
\usepackage{graphicx}
\makeatletter
\def\maxwidth{\ifdim\Gin@nat@width>\linewidth\linewidth\else\Gin@nat@width\fi}
\def\maxheight{\ifdim\Gin@nat@height>\textheight\textheight\else\Gin@nat@height\fi}
\makeatother
% Scale images if necessary, so that they will not overflow the page
% margins by default, and it is still possible to overwrite the defaults
% using explicit options in \includegraphics[width, height, ...]{}
\setkeys{Gin}{width=\maxwidth,height=\maxheight,keepaspectratio}
% Set default figure placement to htbp
\makeatletter
\def\fps@figure{htbp}
\makeatother

\addtolength{\oddsidemargin}{-.5in}%
\addtolength{\evensidemargin}{-1in}%
\addtolength{\textwidth}{1in}%
\addtolength{\textheight}{1.7in}%
\addtolength{\topmargin}{-1in}%
\makeatletter
\makeatother
\makeatletter
\makeatother
\makeatletter
\@ifpackageloaded{caption}{}{\usepackage{caption}}
\AtBeginDocument{%
\ifdefined\contentsname
  \renewcommand*\contentsname{Table of contents}
\else
  \newcommand\contentsname{Table of contents}
\fi
\ifdefined\listfigurename
  \renewcommand*\listfigurename{List of Figures}
\else
  \newcommand\listfigurename{List of Figures}
\fi
\ifdefined\listtablename
  \renewcommand*\listtablename{List of Tables}
\else
  \newcommand\listtablename{List of Tables}
\fi
\ifdefined\figurename
  \renewcommand*\figurename{Figure}
\else
  \newcommand\figurename{Figure}
\fi
\ifdefined\tablename
  \renewcommand*\tablename{Table}
\else
  \newcommand\tablename{Table}
\fi
}
\@ifpackageloaded{float}{}{\usepackage{float}}
\floatstyle{ruled}
\@ifundefined{c@chapter}{\newfloat{codelisting}{h}{lop}}{\newfloat{codelisting}{h}{lop}[chapter]}
\floatname{codelisting}{Listing}
\newcommand*\listoflistings{\listof{codelisting}{List of Listings}}
\makeatother
\makeatletter
\@ifpackageloaded{caption}{}{\usepackage{caption}}
\@ifpackageloaded{subcaption}{}{\usepackage{subcaption}}
\makeatother
\makeatletter
\@ifpackageloaded{tcolorbox}{}{\usepackage[skins,breakable]{tcolorbox}}
\makeatother
\makeatletter
\@ifundefined{shadecolor}{\definecolor{shadecolor}{rgb}{.97, .97, .97}}
\makeatother
\makeatletter
\makeatother
\makeatletter
\makeatother
\ifLuaTeX
  \usepackage{selnolig}  % disable illegal ligatures
\fi
\usepackage[]{natbib}
\bibliographystyle{agsm}
\IfFileExists{bookmark.sty}{\usepackage{bookmark}}{\usepackage{hyperref}}
\IfFileExists{xurl.sty}{\usepackage{xurl}}{} % add URL line breaks if available
\urlstyle{same} % disable monospaced font for URLs
\hypersetup{
  pdftitle={Political Leadership Survival in the Aftermath of Coups or Overstays: From Illegitimate Ascent to Unexpected Exit},
  pdfauthor={Zhu Qi},
  pdfkeywords={Political survival, Coups, Overstays},
  colorlinks=true,
  linkcolor={blue},
  filecolor={Maroon},
  citecolor={Blue},
  urlcolor={Blue},
  pdfcreator={LaTeX via pandoc}}


\begin{document}


\def\spacingset#1{\renewcommand{\baselinestretch}%
{#1}\small\normalsize} \spacingset{1}


%%%%%%%%%%%%%%%%%%%%%%%%%%%%%%%%%%%%%%%%%%%%%%%%%%%%%%%%%%%%%%%%%%%%%%%%%%%%%%

\date{November 19, 2023}
\title{\bf Political Leadership Survival in the Aftermath of Coups or
Overstays: From Illegitimate Ascent to Unexpected Exit}
\author{
Zhu Qi\\
Department of Government, University of Essex\\
}
\maketitle

\bigskip
\bigskip
\begin{abstract}
Previous research predominantly focused on the disruption of regular
leadership tenures by unexpected events, such as coups, self-coups,
uprisings, rebellions, civil wars, or revolutions. In contrast, this
study aims to delve into the longevity of leaders who ascend to power
through these very unexpected events, specifically coups or overstays.
The central argument posits that the endurance of political leadership
is influenced not only by their actions and policies in office but also
by the means through which they come to power. Employing a survival
model, this research investigates the disparities in survival rates
between leaders who rise to power via coups and those who overstay their
terms, and seeks to elucidate the underlying reasons for these
differences.
\end{abstract}

\noindent%
{\it Keywords:} Political survival, Coups, Overstays
\vfill

\newpage
\spacingset{1.9} % DON'T change the spacing!
\ifdefined\Shaded\renewenvironment{Shaded}{\begin{tcolorbox}[frame hidden, sharp corners, boxrule=0pt, borderline west={3pt}{0pt}{shadecolor}, interior hidden, breakable, enhanced]}{\end{tcolorbox}}\fi

\setstretch{1.75}
\hypertarget{introduction}{%
\section{Introduction}\label{introduction}}

The question of why some leaders maintain their hold on power for
extended periods, spanning decades, while others witness their
leadership cut short after mere years, months, or even days, has
captivated scholars and researchers in the field of political science.
This inquiry has been extensively explored in numerous works, as
evidenced by notable contributions such as those by
\citet{clinton1975politics} and \citet{buenodemesquita2003}.

In their seminal work, Bueno de Mesquita and his colleagues undertake a
comprehensive examination of leaders across a diverse political
landscape, encompassing democracies and autocracies, parliamentary and
presidential systems, and both civilian and military contexts. However,
it is worth noting that a significant number of political leaders,
especially in democratic countries, adhere to regular and predictable
tenures. An illustrative example can be found in the United States,
where presidents may serve up to eight years if they secure a second
term, even in cases of suboptimal performance. Similarly, in autocratic
Mexico from 1919 to 2000, each president served a fixed six-year term
without facing overthrows or overstays. In such contexts, the
investigation of tenure length is of marginal significance, as power
transitions between leaders typically occur within the established
framework of constitutional rules or unwritten conventions.

Given the distinctive nature of political leader survival in different
regimes, scholars have increasingly focused on the unexpected tenures,
namely those leaders who do not complete their original terms or those
who overstay their mandates. This shift in focus stems from the fact
that some leaders are toppled by coups, uprisings, rebellions, civil
wars, or revolutions, while others successfully navigate lawful or
unlawful challenges. Previous research on the longevity of political
leaders predominantly centers on two primary dimensions. The first
dimension encompasses the contextual conditions and resources available
to leaders, including factors such as their personal competence
\citep{yu2016}, the stability of their society \citep{arriola2009},
economic performance \citep[\citet{williams2011}]{palmer1999}, access to
natural resources \citep{smith2004, quirozflores2012}, and external
support networks \citep[\citet{thyne2017}]{licht2009, wright2008}. The
second dimension delves into the strategies employed by leaders in
enacting their political and economic policies
\citep{gandhi2007, morrison2009}, as well as their responses to
challenges and dissent within their regimes
\citep{escribà-folch2013, davenport2021}.

Unsurprisingly, a substantial portion of the existing research on
political survival predominantly centers on coups, as they represent the
most common pathways to the exit of authoritarian leaders
\citep{svolik2008, frantz2016}. Previous literature has primarily delved
into the survival of leadership in terms of strategies aimed at
preventing coups \citep{powell2017, sudduth2017, debruin2020}, or how
leaders can extend their tenures after surviving failed coup attempts
\citep{easton2018}.

However, on one hand, the duration of political leaders' tenures can be
significantly influenced by unforeseen events like coups. On the other
hand, these very unexpected events that usher in new leaders can also
become the catalyst for the subsequent cycle of unexpected developments.
It is conceivable that leaders who come to power through regular and
constitutional transitions are more likely to undergo periodic shifts in
leadership, while those who seize power through unconstitutional means
face a higher risk of unanticipated removal from office. Unfortunately,
there has been a limited emphasis on the study of leadership survival in
the context of successful coups. A similar research gap exists in the
examination of incumbents who overstay their terms in power, which forms
the central focus of this paper.

The analysis of their tenures holds particular significance for two
reasons. Firstly, the duration of these leaders' tenures exhibits
considerable variation, ranging from mere months to several decades.
Secondly, predicting the tenures of such leaders proves challenging. A
seemingly robust and stable regime can collapse suddenly overnight,
while an apparently fragile one might persist for decades. The
substantial disparities in these tenures remain inadequately explained,
posing a perplexing challenge that has piqued the interest of numerous
political scientists.

Expanding on the discourse surrounding coups and leaders who overstay
their intended terms, this paper delves into the trajectories of
political leaders who came to power through coups or extended their
mandates beyond the originally intended tenure. The central objective is
to examine the variations in survival duration between leaders who
attain power through coups and those who exceed their terms, while also
shedding light on the underlying factors contributing to these
distinctions.

This paper follows a structured approach as outlined below: The second
section encompasses a comprehensive literature review on political
survival and highlighting the contributions of this paper might offer.
The third chapter delves into the examination of factors influencing the
survival of leaders who have ascended to power through unconstitutional
means. Chapter 4 provides an account of the methodology and data
employed, utilizing a survival model for a comprehensive analysis of the
determinants of leaders' survival. The subsequent chapter, Chapter 5,
presents the findings of this analysis, facilitating an in-depth
discussion of the results. Finally, in Chapter 6, the paper concludes by
synthesizing these findings and exploring their broader implications.

\hypertarget{literature-review-the-dynamics-of-leadership-survival-in-different-scenarios}{%
\section{Literature review: The dynamics of leadership survival in
different
scenarios}\label{literature-review-the-dynamics-of-leadership-survival-in-different-scenarios}}

In their ambitious work, \citet{buenodemesquita2003} set out to provide
a comprehensive explanation for the logic of political leadership
survival within a universal framework. According to this framework, the
endurance of political leaders hinges on the maintenance of a supportive
winning coalition. However, it is essential to notice that the dynamics
of leadership survival differ significantly across various types of
regimes.

In democratic systems, distinctions emerge between parliamentary and
presidential regimes. For instance, in parliamentary countries such as
the UK and Japan, political parties may maintain power for extended
periods, even as prime ministers change frequently. A contemporary
example is the United Kingdom in 2022, which saw three different prime
ministers while the Conservative Party retained its grip on power. In
contrast, in presidential countries like the United States, leaders are
subject to fixed terms, making power transitions more regular and
predictable.

Moreover, the concept of dividing the electorate into a selectorate and
a winning coalition may not be as relevant in democracies. While those
who support and vote for incumbents may witness their preferred policies
enacted, those who vote against them still experience the same policies.
For instance, individuals who cast their votes for candidates advocating
lower taxes face the same tax rates as those who vote against the
incumbents. This doesn't result in lower taxes for supporters and higher
taxes for opponents; rather, both groups encounter identical tax levels.
Consequently, it becomes challenging to argue that winning coalitions
inherently gain a significant advantage over the broader selectorate in
democratic systems.

On the other hand, there are types of autocratic regimes, each with its
distinct characteristics, including civilian autocracy, personnel
autocracy, military regimes, party dominance, and monarchies. In most
autocratic regimes, the process of leadership selection remains veiled
in secrecy. For instance, in countries like China, the mechanisms for
appointing leaders often resemble a black box, concealing the rules and
procedures from outsiders. Expressing dissenting views, whether as
potential challengers or supporters of challengers, can be perilous. In
Russia, despite the presence of general elections, challengers
frequently face severe consequences such as assassination, poisoning,
imprisonment, or exile. As a result of the absence of transparent and
fair conventional procedures for power transitions, leaders in
autocratic regimes are more vulnerable to being deposed through coups or
other unconventional means.

Beyond distinctions among various regimes, the endurance of leaders can
fluctuate even within the confines of a single regime, contingent on the
circumstances they encounter. It stands to reason that leaders ascending
through conventional means may experience a different survival
trajectory compared to those ushered in by coups or those who overstay
their terms. Additionally, leaders operating in favorable economic,
social, and international contexts are likely to have a more prolonged
tenure compared to their counterparts navigating challenging conditions.

Considering the factors discussed earlier, a substantial portion of
existing literature seeks to unravel the underlying principles governing
political survival in non-democratic regimes. Notably, scholarly
attention has gravitated towards the examination of coup-proofing
strategies, given that coups emerge as a primary driver of irregular
exits in autocracies
\citep{quinlivan1999, powell2014, sudduth2017, tang2021}. Additionally,
there is a notable focus on the study of survival strategies following
failed coup attempts, as evidenced by the works of
\citep{kebschull1994, easton2018, oztig2020}.

In \citet{sudduth2017a}, the author delves into the post-coup actions of
a dictator, despite the primary focus of the paper being on purge
strategies. The central argument asserts that leaders who rise to power
through coups experience a temporary surge in influence compared to the
elites immediately following the coups, making them less susceptible to
being ousted by subsequent coup attempts. This assertion, as highlighted
by the author, challenges the conventional notion that new leaders are
generally in a position of weakness in the initial stages of their
tenure \citep{roessler2011}. Regardless of their initial strength, both
Sudduth and previous scholars concur that new leaders are inclined to
purge rival elite groups to bolster their power. The distinction lies in
Sudduth's claim that dictators undertake purges when they possess the
capability to do so without significant risk, while conventional views
posit that dictators resort to purges when compelled to prevent
potential ousting by coups.

Yet, it's important to recognize that new leaders, especially those who
ascend through unconventional means, don't conform to a universal
pattern of being either inherently weak or powerful. Leadership
transitions occur in diverse contexts, and thus, leaders face a spectrum
of challenges. Some emerge in positions of vulnerability, while others
wield considerable strength. Regardless of individual power, when
juxtaposed with the entirety of elites or the entire population, leaders
remain in a position of relative weakness---unity among elites or
residents can overshadow even the most powerful leaders.

\hypertarget{the-logic-of-political-leader-survival-in-irregular-ascensions}{%
\section{The logic of political leader survival in irregular
ascensions}\label{the-logic-of-political-leader-survival-in-irregular-ascensions}}

Engaging in a discourse on the survival strategies of political leaders
within non-democratic regimes presents a significant challenge. The
complexity stems from the lack of a universal pattern that encapsulates
the rules or conventions dictating power transitions in autocratic
systems. For instance, even in Middle Eastern monarchies, the transfer
of power doesn't rigidly adhere to a father-to-son lineage. However,
this doesn't imply that analyzing survival strategies in autocracies is
unattainable. Despite substantial differences, they share certain
commonalities. Most autocratic regimes, especially those characterized
by irregular ascensions, exhibit three prevalent situations.

The first aspect concerns the issue of legitimacy. Leaders who ascend
through coups lack legitimacy as they seize power through force or other
unconventional means. While many leaders prolong their tenures through a
façade of constitutional procedures, such as judgments by the Supreme
Court, congressional votes, or even referendums, they often manipulate
these processes to maintain control. It's commonly understood by ruling
elites, opposition parties, and the populace that these leaders lack
legitimacy. This lack of legitimacy can sometimes justify the cause of
those seeking their replacement, even if the means used are
unconstitutional.

The second characteristic revolves around the uncertainty surrounding
power transitions. This uncertainty creates ambiguity not only for
ruling elites and ordinary citizens but also for the leaders themselves
regarding when, how, and to whom power might be transferred. Such
uncertainty breeds inherent instability. Amidst such instability, people
experience a lack of security. This perception often leads to the belief
that the current ruler is incompetent and should be replaced by someone
more powerful or capable. Consequently, the ruling elite or opposition
factions may exploit the instability as an opportunity to challenge
existing power structures.

The third aspect involves the collective action problem within the
opposition. As highlighted in the introduction, expressing dissenting
views in autocratic systems, either as potential challengers or
supporters of challengers, can be perilous. The absence of free public
expression renders the power balance unclear. Beyond small, close-knit
groups, it becomes impossible to discern who supports the incumbent and
who opposes them. Attempts to determine the numbers of supporters or
opponents and efforts to unite different anti-incumbent factions can
also be risky. Consequently, anti-incumbent forces often maintain a low
profile for safety until a particular group gains the confidence to
succeed.

The trifecta of illegitimacy, uncertainty, and collective action issues
profoundly impact the longevity of a regime. Yet, compared to leaders
who gain power through coups, those who overstay their terms find
themselves in a comparatively advantageous position concerning these
three aspects.

\hypertarget{legitimacy}{%
\subsection{Legitimacy}\label{legitimacy}}

As per Powell and Thyne's definition, coups constitute ``illegal and
overt attempts by the military or other elites within the state
apparatus to unseat the sitting executive.'' \citep[p.252]{powell2011}.
While it is undeniable that a few coups have been justified by resolving
crises and leading to improved outcomes, they remain illegal means to
remove incumbents. These unlawful methods open Pandora's box, publicly
suggesting alternatives to constitutional procedures for seizing power,
particularly in the case of successful coups. Such actions inevitably
prompt imitators to launch new coups. As society becomes accustomed to
coups, subsequent ones may not elicit significant backlash, given that
the incumbents themselves ascended to power through similar means.
Furthermore, coups not only invite further coups but also embolden
external challengers, including uprisings, revolutions, and civil wars
\citep{dahl2023}.

On the other hand, leaders who overstay their tenures may lack
legitimacy but often manage to maintain power through a facade of
legitimacy. They don't blatantly seize power via military force but
rather cling to power through parliamentary or congressional processes,
the Supreme Court, and even nationwide referendums. The opposition
usually chooses to confront these leaders using legal means, engaging in
legislative debates or legal proceedings, and sometimes by advocating
for another referendum. Attempts to overthrow leaders who overstay their
terms through coups would be even less legitimate and might struggle to
garner support. However, removing such leaders within the boundaries of
the law presents an arduous challenge, if not a near-impossible task.

\hypertarget{uncertainty-and-instability}{%
\subsection{Uncertainty and
instability}\label{uncertainty-and-instability}}

When discussing instability, it's clear that regimes following coups are
more prone to instability. In many cases, coups occur amidst chaotic
situations, with coup leaders using the pretext of restoring order. To
expedite the restoration of order, leaders emerging from coups may find
themselves compelled to make compromises among various internal or
external factions. As noted by \citet{roessler2011}, these rulers might
try to reduce the likelihood of further coups at the cost of increasing
the risk of societal rebellions and civil wars. This is also the logic
of the Chinese ruler, Chiang Kai-shek, in 1930s. In order to eliminate
the internal threats, Chinese Communist Party, he resorted to a strategy
of compromise with both Japan and the Soviet Union. ``Domestic stability
takes precedence over resisting foreign invasion'' was one of the most
commonly used slogans by Chiang Kai-shek during that time
\citet{chu1999chiang}.

Survival in power relies significantly on the cohesion of the ruling
group. As numerous scholars have pointed out, internal conflicts among
elites pose a more serious threat to the stability of those in power.
Coups often lay bare the fractures within a regime, not only attracting
more followers to orchestrate new coups but also emboldening external
challengers, including uprisings, revolutions, and civil wars. On the
other hand, successful tenures unmistakably showcase the incumbents'
firm grasp on power, discouraging both internal dissent and external
threats

For leaders who ascend to power through coups or overstay their tenure,
these conflicts typify their rules. Following coups, regimes face four
potential outcomes: democracies initially overthrown by coups may either
persist as democracies (I) or transform into autocracies (II), while
autocracies overthrown by coups may either endure as autocracies (III)
or evolve into democracies (IV). However, even in democracies reinstated
after coups, a degree of uncertainty lingers. In contrast, regimes
characterized by leaders overstaying their terms, with only rare
exceptions, typically persist as autocracies or undergo a transition
towards autocracy.

In an environment characterized by uncertainty, the primary strategy for
rulers is to maintain the stability of society. The Chinese Communist
Party, for example, consistently emphasizes in public statements that
`\textbf{Stability is Everything}'. The underlying logic is clear:

In stark contrast, leaders who attain power through coups confront more
formidable initial challenges than those who overstay their terms. While
rulers exceeding their terms might incite discontent and unrest,
successful tenures unmistakably demonstrate the incumbents' robust
control over power, effectively discouraging both internal dissent and
external threats. This, in turn, contributes to the overall stability of
the governing structure and society. Consequently, the pressing need to
dismantle the old ruling paradigm and establish a new order is markedly
diminished in this context.

On the contrary, coup-instated leaders almost always ascend to power
amid instability, even if some coups are executed peacefully. They
overthrow previous rulers, necessitating the dismantling of at least
part of the old ruling framework and the replacement of certain officers
and officials. These actions inevitably lead to turbulence and create
adversaries for the new rulers. Distributing the benefits of successful
coups among supporters poses another challenge, as it is nearly
impossible to satisfy every supporter. If grievances, whether from
former ruler supporters or the coup-instated leaders' backers,
intensify, the new leaders might resort to purging some of them, further
fueling chaos.

Moreover, despite temporarily wielding power, leaders who come to office
through coups face enduring challenges. Coups often expose the fractures
within a regime, not only galvanizing more adherents to plot subsequent
coups but also emboldening external adversaries, including uprisings,
revolutions, and civil wars. A comprehensive coup dataset
\citep{powell2011} spanning from 1950 to 2023 reveals that 97 countries
experienced coups during this period. Among them, 15 countries endured
at least 10 coups, and 10 countries witnessed more than 6 successful
coups. These factors collectively contribute to a decrease in the
expected lifespan of the regime after coups \citep{dahl2023}.

\textbf{Hypothesis:} Political leaders who successfully extend their
time in power are more likely to have prolonged survival compared to
leaders who assume power through coups.

\hypertarget{method-and-data}{%
\section{Method and data}\label{method-and-data}}

\newpage


\renewcommand\refname{References}
  \bibliography{survival.bib}


\end{document}
